\documentclass[]{article}
\usepackage{blindtext}
\usepackage{hyperref}
\author{Marcin Tomczyk}
\date{23.10.2018}
\begin{document}
\indent %wstawia odstęp poziomy równy wcięciu na %początku akapitu


Test
\noindent
\begin{center}
Test2			%wyśrodkowany napis
\end{center}
 
\begin{enumerate}
\item wyliczenie 1
\item wyliczenie 2
\end{enumerate}
%komentarz
%tworzymy spis tresci
\tableofcontents



\section{sekcja1}
\blindtext %sluzy do przejrzenia losowego tekstu
\begin{tabular}{||c|r||} \hline		%tworzymy tabele
\multicolumn{2}{||c||}{Moj pierwszy wiersz}
\\ \hline \hline
(1,1) & (1,2) \\
(2,1) & (2,2) \\ \hline
(3,1) & (3,2) \\
(4,1) & (4,2) \\ \hline
\end{tabular}

\footnotemark %tzw "odsylacz"
\section{sekcja2}
\blindtext

\section{sekcja3}
\blindtext

\footnote[1]{to jest przypis}
\section{sekcja4}
\blindtext

\begin{thebibliography}{9}
\bibitem {} Mikolaj Kopernik. 
\textit{O obrotach cial niebieskich}. 
 Torun, 1530.
 
 \bibitem {} J.R.R.Tolkien. 
\textit{Hobbit}. 
 Londyn, 1947.


\end{thebibliography}



\end{document}